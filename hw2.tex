\documentclass{article}
\usepackage{geometry}
\usepackage{amsmath}
\usepackage{amssymb}
\usepackage{multirow}
\usepackage{fancyhdr}
\pagestyle{fancy}

\lhead{MATH381.004 --- Homework 2}
\rhead{\textbf{Jason He}}

\begin{document}
\begin{enumerate}
    \item[{[\S 1.4]} 2a.] \fbox{TRUE.} $x = \texttt{orange}$ contains the letter \texttt{a} in its third position from the left.
    \item[3b.] \fbox{FALSE.} $x = \textrm{Detroit}$ is not the capital of $y = \textrm{Michigan}$. The capital is Lansing.
    \item[5.]
        \begin{itemize}
            \item[(a)] ``There exists a student who spends more than five hours every weekday in class.''
            \item[(b)] ``All students spend more than five hours every weekday in class.''
        \end{itemize}
    \item[9.]
        \begin{itemize}
            \item[(a)] $\exists x \, \left( P(x) \land Q(x) \right)$
            \item[(c)] $\forall x \, \left( P(x) \lor Q(x) \right)$
        \end{itemize}
    \item[13.]
        \begin{itemize}
            \item[(a)] \fbox{TRUE.} Any integer, plus one, is strictly greater than the integer by itself.
            \item[(c)] \fbox{TRUE.} Namely, if $n = 0$, then $-0 = 0$ is a true statement.
        \end{itemize}
    \item[16.]
        \begin{itemize}
            \item[(b)] \fbox{FALSE.} The square of any real number is nonnegative.
            \item[(d)] \fbox{FALSE.} $x = 0$ is a counterexample; $0^2 = 0$.
        \end{itemize}
    \item[17d.] $\neg P(0) \land \neg P(1) \land \neg P(2) \land \neg P(3) \land \neg P(4)$
    \item[21.]
        \begin{itemize}
            \item[(c)]
                \begin{itemize}
                    \item If the domain consists of \underline{Donald Trump Jr., Ivanka Trump, and Eric Trump}, then the statement is true, because all three have the same mother, namely Ivana Trump, and so any combination of two people in this domain will have the same mother.
                    \item If the domain consists of \underline{Ted Cruz, Marco Rubio, and John Kasich}, then the statement is false because all have different mothers.
                \end{itemize}
            \item[(d)]
                \begin{itemize}
                    \item If the domain consists of \underline{Hillary Clinton, Bernie Sanders, and Martin O'Malley}, then the statement is true because all have different grandmothers.
                    \item If the domain consists of \underline{Sasha Obama, Malia Obama, and Chelsea Clinton}, then the statement is false because Marian Robinson is the grandmother of both Sasha and Malia, who are two different people.
                \end{itemize}
        \end{itemize}
    \item[25.] Let the domain be all people, and
        \begin{align*}
        P(x) &= \textrm{``$x$ is perfect''}\\
        Q(x) &= \textrm{``$x$ is your friend''}
        \end{align*}
        Then
        \begin{itemize}
            \item[(a)] $\forall x \, \neg P(x)$
            \item[(d)] $\exists x \, \left( Q(x) \land P(x) \right)$
        \end{itemize}
    \item[35b.] $x = 0$ is a counterexample. $0 \ngtr 0$ and $0 \nless 0$.
    \item[40b.] Let
        \begin{align*}
        P(x) &= \textrm{``$x$ can be opened''} \tag*{\textit{domain:} directories in the file system}\\
        Q(y) &= \textrm{``$y$ can be closed''} \tag*{\textit{domain:} files}\\
        R &= \textrm{``System errors have been detected''}
        \end{align*}
    Then \fbox{$R \rightarrow \forall x \, \forall y \, \left( \neg P(x) \land \neg Q(y) \right)$}.

    % If \emph{multiple} system errors are required (the question is not clear) then let
    % \[
    % S = \textrm{``System errors have been detected''}
    % \]
    % so the expression is $S \rightarrow \forall x \, \forall y \, \left( \neg P(x) \land \neg Q(y) \right)$.
    \item[45.] As shown in the January 23 lecture, there are three cases to consider.
        \begin{itemize}
            \item \underline{Case 1.} There exists some $x$ for which $P(x)$ is true.
            \begin{itemize}
                \item[$\Rightarrow$] $\exists x \, P(x)$ is true $\Rightarrow \exists x \, P(x) \lor \exists x \, Q(x)$ is true.
                \item[$\Rightarrow$] $P(x) \lor Q(x)$ is true $\Rightarrow \exists x \, \left( P(x) \lor Q(x) \right)$ is true.
            \end{itemize}
            \item \underline{Case 2.} There exists some $x$ for which $Q(x)$ is true.
            \begin{itemize}
                \item[$\Rightarrow$] $\exists x \, Q(x)$ is true $\Rightarrow \exists x \, P(x) \lor \exists x \, Q(x)$ is true.
                \item[$\Rightarrow$] $P(x) \lor Q(x)$ is true $\Rightarrow \exists x \, \left( P(x) \lor Q(x) \right)$ is true.
            \end{itemize}
            \item \underline{Case 3.} For all $x$, $P(x)$ and $Q(x)$ are both false.
            \begin{itemize}
                \item[$\Rightarrow$] $\exists x \, P(x)$ is false. $\exists x \, Q(x)$ is false. By implication, $\exists x \, P(x) \lor \exists x \, Q(x)$ is false.
                \item[$\Rightarrow$] $P(x) \lor Q(x)$ is false $\Rightarrow \exists x \, \left( P(x) \lor Q(x) \right)$ is false.
            \end{itemize}
        \end{itemize}
    Therefore it is proven that
    \[
    \exists x \, \left( P(x) \lor Q(x) \right) \Longleftrightarrow \exists x \, P(x) \lor \exists x \, Q(x)
    \]
    \item[{[\S 1.5]} 2.]
        \begin{itemize}
            \item[(a)] ``There exists a real number $x$ such that for every real number $y$, the product of $x$ and $y$ equals $y$.''
            \item[(c)] ``For every real number $x$ and every real number $y$, there exists a real number $z$ such that $x$ equals the sum of $y$ and $z$.''
        \end{itemize}
    \item[3.]
        \begin{itemize}
            \item[(a)] ``There exists a student $x$ and a student $y$ such that $x$ has sent an e-mail message to $y$.''
            \item[(c)] ``For every student $x$, there exists a student $y$ such that $x$ has sent an e-mail message to $y$.''
                % \begin{itemize}
                %     \item More intuitively: \emph{every student has emailed at least one student.}
                % \end{itemize}
        \end{itemize}
    \item[8.]
        \begin{itemize}
            \item[(a)] $\exists x \, \exists y \, Q(x,y)$
            \item[(b)] $\forall x \, \forall y \, \neg Q(x,y)$
            \item[(c)] $\exists x \, \left( Q(x, \textrm{Jeopardy}) \land Q(x, \textrm{Wheel of Fortune}) \right)$
        \end{itemize}
    \item[23.]
        \begin{itemize}
            \item[(a)] $\forall x \, \forall y \, \left( \left( x<0 \land y<0 \right) \rightarrow xy>0 \right)$
            \item[(c)] $\forall x \, \left( x > 0 \rightarrow \exists a \, \exists b \, \left( (a \neq b) \land \forall c \, \left( c^2 = x \leftrightarrow c = a \lor c = b \right) \right) \right)$
        \end{itemize}
        where the domain for all variables consists of all real numbers.
    \item[26.]
        \begin{itemize}
            \item[(e)] \fbox{TRUE.} Let $x = y = 0$; $0 + 0 = 0 - 0$.
            \item[(f)] \fbox{TRUE.} Let $y = 0$; $x + 0 = x - 0$ for all $x$.
        \end{itemize}
    \item[31.]
        \begin{itemize}
            \item[(b)] By De Morgan's laws,
                \begin{align*}
                \neg \left( \forall x \, \exists y \, P(x,y) \lor \forall x \, \exists y \, Q(x,y) \right) &\equiv \neg \forall x \, \exists y \, P(x,y) \land \neg \forall x \, \exists y \, Q(x,y) \\%\tag*{by De Morgan's laws}\\
                &\equiv \exists x \, \neg \exists y \, P(x,y) \land \exists x \, \neg \exists y \, Q(x,y) \\%\tag*{by De Morgan's laws for quantifiers}\\
                &\equiv \exists x \, \forall y \, \neg P(x,y) \land \exists x \, \forall y \, \neg Q(x,y) %\tag*{by De Morgan's laws for quantifiers}
                \end{align*}
            \item[(c)] Again by De Morgan's laws,
                \begin{align*}
                \neg \forall x \, \exists y \, \left( P(x,y) \land \exists z \, R(x,y,z) \right) &\equiv \exists x \, \neg \exists y \, \left( P(x,y) \land \exists z \, R(x,y,z) \right) \\
                &\equiv \exists x \, \forall y \, \neg \left( P(x,y) \land \exists z \, R(x,y,z) \right) \\
                &\equiv \exists x \, \forall y \, \left( \neg P(x,y) \lor \neg \exists z \, R(x,y,z) \right) \\
                &\equiv \exists x \, \forall y \, \left( \neg P(x,y) \lor \forall z \, \neg R(x,y,z) \right)
                \end{align*}
        \end{itemize}
\end{enumerate}
\end{document}