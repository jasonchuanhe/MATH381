\documentclass{article}
\usepackage{geometry}
\usepackage{amsmath}
\usepackage{amssymb}
\usepackage{multirow}
\usepackage{fancyhdr}
\pagestyle{fancy}

\lhead{MATH381.004 --- Homework 11}
\rhead{\textbf{Jason He}}

% 6.3:  28, 33,  37

% 6.4:   2,4, 7, 12, 15, 19

% 6.5:   1, 3, 6, 8, 11, 17 

\begin{document}

%\textsc{Notation.} $\displaystyle\binom{n}{m}$ is the binomial coefficient.

\begin{enumerate}
    \item[{[\S 6.3]} 28.] $\displaystyle\binom{40}{17} = 88{,}732{,}378{,}800$ keys
    \item[33.] $\displaystyle\binom{10}{3}\binom{15}{3} = 54{,}600$ committees
    \item[37.] Length-10 bit strings of at least three of both \textcircled{\texttt{1}} and \textcircled{\texttt{0}} must be comprised of either
        \begin{itemize}
            \item $3\times\textcircled{\texttt{1}}$ and $7\times\textcircled{\texttt{0}}$,
            \item $4\times\textcircled{\texttt{1}}$ and $6\times\textcircled{\texttt{0}}$,
            \item $5\times\textcircled{\texttt{1}}$ and $5\times\textcircled{\texttt{0}}$,
            \item $6\times\textcircled{\texttt{1}}$ and $4\times\textcircled{\texttt{0}}$, or
            \item $7\times\textcircled{\texttt{1}}$ and $3\times\textcircled{\texttt{0}}$;
        \end{itemize}
    thus there are $\displaystyle\sum_{3 \le k \le 7} \binom{10}{k} = \fbox{912 strings}$.
    \item[{[\S 6.4]} 2.]
        \begin{itemize}
            \item[(a)] $(x+y)^5 = (x+y)(x+y)(x+y)(x+y)(x+y)$ gives rise to six terms:
                \begin{itemize}
                    \item[i.] $x^5$, obtainable $\binom{5}{5} = 1$ way (viz.~$xxxxx$)
                    \item[ii.] $x^4 y$, obtainable $\binom{5}{4} = 5$ ways ($xxxxy, xxxyx, \ldots$)
                    \item[iii.] $x^3 y^2$, obtainable $\binom{5}{3} = 10$ ways
                    \item[iv.] $x^2 y^3$, obtainable $\binom{5}{2} = 10$ ways
                    \item[v.] $x y^4$, obtainable $\binom{5}{1} = 5$ ways
                    \item[vi.] $y^5$, obtainable $\binom{5}{0} = 1$ way
                \end{itemize}
            Therefore, $(x+y)^5 = x^5 + 5 x^4 y + 10 x^3 y^2 + 10 x^2 y^3 + 5 x y^4 + y^5$.
            
            \item[(b)] Using the binomial theorem,
            \begin{align*}
            (x+y)^5 &= \sum_{0 \le j \le 5} \binom{5}{j} x^{5-j}y^j \\
            &= \binom{5}{0} x^5 + \binom{5}{1} x^4 y + \binom{5}{2} x^3 y^2 + \binom{5}{3} x^2 y^3 + \binom{5}{4} x y^4 + \binom{5}{5} y^5 \\
            &= x^5 + 5 x^4 y + 10 x^3 y^2 + 10 x^2 y^3 + 5 x y^4 + y^5.
            \end{align*}
        \end{itemize}
    \item[4.] The coefficient of $x^5 y^8$ is $\displaystyle\binom{13}{8} = \fbox{1287}$.
    \item[7.] The coefficient of $x^9$ is $\displaystyle (-1) \binom{19}{9} 2^{(19-9)} = \fbox{$-94{,}595{,}072$}$.
    \item[12.] Row 11 is \fbox{$1 \quad 11 \quad 55 \quad 165 \quad 330 \quad 462 \quad 462 \quad 330 \quad 165 \quad 55 \quad 11 \quad 1$}.
    \item[15.] Using the binomial theorem with $x = y = 1$, $2^n$ can be expressed as
    \[
    (1+1)^n = \sum_{0 \le k \le n} \binom{n}{k} 1^k 1^{n-k} = \sum_{0 \le k \le n} \binom{n}{k},
    \]
    and since $\left(\sum \binom{n}{k}\right) \ge \binom{n}{k}$, it follows that $\binom{n}{k} \le 2^n$. $\square$
    \item[19.] Pascal's Identity is
        \begin{align*}
        \binom{n}{r-1} + \binom{n}{r} &= \frac{n!}{(r-1)!(n-(r-1))!} + \frac{n!}{r!(n-r)!} \\
        &= \frac{n!(r+(n-r+1))}{r!(n-r+1)!} \\
        &= \frac{(n+1)!}{r!(n+1-r)!} \\
        &= \binom{n+1}{r}. \tag*{$\square$}
        \end{align*}
    \item[{[\S 6.5]} 1.] $3^5 = 243$
    \item[3.] $26^6 = 308,915,776$
    \item[6.] $\displaystyle\binom{3+5-1}{5} = 21$
    \item[8.] $\displaystyle\binom{21+12-1}{12} = 225{,}792{,}840$
    \item[11.] The number of pennies and nickels is irrelevant as long as there are at least eight of each. There are $\displaystyle\binom{2+8-1}{8} = \fbox{9 ways to choose eight coins}$.
    \item[17.] $\dfrac{10!}{2!3!5!} = 2{,}520$ strings
\end{enumerate}
\end{document}